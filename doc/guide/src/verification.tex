\chapter{Running simulations}

Available tests are described in detail in Chapter~4 of \cite{friedmann13phd}.
Here, we will focus on how to use theses tests.
In order to do that, we will also discuss the Makefile-based build flow.


\section{Build flow}
The build flow uses \code{make} and was inspired by the Linux build system kbuild \citep{kbuild}.
There is a central configuration file \file{.config} in the top-level directory that defines configuration options.
Source files are collected by using \file{Makefile.srclist} files throughout the directory hierarchy using the configured options.
Separate files \code{Makefile.build} hold the build rules for \code{make} for each supported simulator.
These Makefiles reside in run directories under \file{verification/sim_*}.
Currently only ModelSim is supported in \file{verification/sim_model/}.
Run directories for synthesis are located in \file{target/}.


\subsection{Configuration options}

These configuration options can be given in the global configuration file \file{.config}.

\begin{description}
    \item[\code{CONFIG_PLATFORM}] \textit{= [~virtex5 | tsmc65 | umc180 | designware | achronix\_speedster22ihd~]}

        Select the target technology.
        Some components for clock generation, memories, multipliers, etc.\ only work in particular technologies, for example when using specialized \gls{fpga} resources.
        Note, that support for the achronix \gls{fpga} is merely a placeholder.

    \item[\code{CONFIG_BOARD}] \textit{= [~ml505 | flyspi-board~]}

        Select which \gls{fpga} board is used.
        The Xilinx evaluation board ml505 \citep{ml505_ug_2008} or the ``flyspi'' board of the electronic vision(s) group.
        Support for the latter is preliminary.
        This option is only relevant for the \file{target/emsys} implementation run directory.

    \item[\code{CONFIG_USE_XILINX_MULTIPLIER}] \textit{= [~y | n~]}

        Select whether to use integrated DSP slices, when building for \code{CONFIG_PLATFORM} = virtex5.

    \item[\code{CONFIG_WITH_BUS}] \textit{= [~y | n~]}

        Build the \gls{omnibus} implementation provided as part of the repository or not.


    \item[\code{CONFIG_WITH_VECTOR}] \textit{= [~y | n~]}

        Build modules for the \gls{fxv} unit or not.
\end{description}


\subsection{Running simulations}

To compile sources for simulation with ModelSim according to the selected configuration do:
\begin{lstlisting}[language=bash]
   $ cd verification/sim_model
   $ make
   $ vsim -do <simulation script>
\end{lstlisting}
Several simulation scripts are provided for the different test scenarios.

\section{Tests}

\subsection{Program test}

\begin{tabular}{ll}
    \textit{Simulation script:}     & \file{verification/sim_model/sim_plt.do} \\
    \textit{Top-level source:}      & \file{testbenches/program_test.sv} \\
\end{tabular} \\
This test executes a number of programs from \file{test/testcode}.
The processor is simulated in a two memory environment.
Therefore, each program provides code and memory images, plus an additional expected data image.
The test compares the memory contents after simulation to the expected image.
Memory images can be provided as ASCII files with hexadecimal values or in raw binary form.
For new programs the latter format should be preferred, since it can be generated using the flow described in Chapter~\ref{ch:usesw}.


\subsection{Sequence test}

\begin{tabular}{ll}
    \textit{Simulation script:}     & \file{verification/sim_model/sim_seq.do} \\
    \textit{Top-level source:}      & \file{testbenches/sequence_test.sv} \\
\end{tabular} \\
This test generates random program sequences and compares the final state of internal registers to the expected state.
\Gls{fxv} instructions are excluded from random generation.
Also the \code{tw} and \code{twi} instructions are not allowed to avoid exceptions.
The test can be configured with four preprocessor options:
\begin{description}
    \item[\code{PROGRAM_LENGTH}]
        Number of instructions of which the last one is always \code{wait}.

    \item[\code{OPT_BCACHE}]
        Passed to the top-level processor option of the same name (see Section~\ref{sec:topopt}).

    \item[\code{USE_CACHE}]
        Perform the test using an instruction cache.

    \item[\code{OPT_DMEM_BUS}]
        If set, use $\code{OPT_DMEM} = \code{Pu_types::DMEM_BUS}$ (see Section~\ref{sec:topopt}).
\end{description}
The simulation itself runs indefinitely.
Specify a time when starting the simulation.
For good coverage a simulated time of \unit[100]{ms} should be aimed for.


\subsection{Vector unit test}
\begin{tabular}{ll}
    \textit{Simulation script:}     & \file{verification/sim_model/sim_fub_vector.do} \\
    \textit{Top-level source:}      & \file{testbenches/fub_vector_test.sv} \\
\end{tabular} \\
This is a unit test for the vector unit.
It tests the design in three phases:
\begin{enumerate}
    \item Explicitly specified programs.
    \item Randomly generated single instructions.
    \item Randomly generated sequences of instructions.
\end{enumerate}
A larger test set is defined by \file{testbenches/signoff_vector_test.sv} that includes the top-level of this test.
The simulation script \file{verification/sim_model/sim_signoff_vector_test.do} uses this as top-level.
